\section{程序亮点与不足}
\subsection{亮点}
\subsubsection{函数封装}
除主函数外,通过以下23个函数实现全部功能,每个函数目标明确,易于阅读和修改。同时在函数调用关系图\ref{fig:code}中可以看到,一个函数被多个函数调用,说明代码复用性好。

\lstset{language=C}
\begin{lstlisting}
void dataInit();
void Menu_plot();
void Menu();
void GameInit();
void ImportImage();
void GamePlay();
int is_connect(int x1, int y1, int x2, int y2,int is_draw);
void state_upd();
int is_in_col(int x1, int y1, int x2, int y2); 
int is_in_row(int x1, int y1, int x2, int y2);
int is_connect_one(int x1, int y1, int x2, int y2); 
int is_connect_two(int x1, int y1, int x2, int y2); 
int is_connect_three(int x1, int y1, int x2, int y2, int is_draw); 
void clear();
int is_satisfy_10(); 
void RankInit();
int is_win();
void mode_choose();
void setting();
void setting_plot();
void rank_plot();
int is_connect_solvable();
void auto_remove();
\end{lstlisting}

\subsubsection{程序保护}
当提示次数和洗牌次数为0时,玩家再次点击相应按钮会提示次数用完,而不会出现负值情况,对程序进行保护,代码如下。
\lstset{language=C}
\begin{lstlisting}
//鼠标点击提示区域
else if  (x > 760 && x < 850 && y >245 && y < 295){
    if (shuffle == 0) {
        MessageBox(GetHWnd(), L"你的提示次数已经用完!", L"提示", 0);
     }
    else {
        //进入正常的提示消除部分
        //省略
    }
}
\end{lstlisting}

\subsubsection{图像刷新}
很多样例程序采用在每次while循环中对全部图片进行刷新,导致出现闪屏的情况。这里为了避免这个情况,在GameInit中一次性显示图片,之后每次消除后只需要将对应清空的格点置白。但这样也给其他功能的实现带来麻烦,例如消去连接提示线(每次成功消去短暂停留的连接线)时,就需要额外的clear\_list记录需要消去的格子。

\subsubsection{充值功能}
充值提示功能原本是我找到提示对后没有break的BUG,但这个BUG被我改成了一个充值亮点。同时这种小游戏的充值也讽刺了当今的游戏环境。


\subsection{不足}
\subsubsection{边界处理}
玩家无法对边界上的相同图片通过边界连线进行消除。这里的实现思路只需要将原MAP周围加上一圈空白格即可。但由于一开始编程时我没注意这点,且程序中对MAP采用了两种如下的索引方式,我尝试修改后没能正确实现。最后向谌老师征求意见,得到答复是这属于功能实现范畴,不属于编程问题,于是我就没对边界格点消除进行实现。

\lstset{language=C}
\begin{lstlisting}
//1维索引
for (int i = 0; i < max_col * max_row  ; i++) 
    MAP[i] = rand() % (num_hero) + 1;
//2维索引
for (int i = 0; i < max_col; i++)
    for (int j = 0; j < max_row; j++) 
        putimage(start_x + i * box_width, start_y + j * box_width, &img[MAP[i + j * max_col]]);
\end{lstlisting}

\subsubsection{提示线绘制}
提示线消除的代码集中在判断可连性处,但由于我只返回是否可连的01变量,因此具体实现复杂,无法复用,例如练练规则改成允许四段连接,程序就需要大改。具体来说一段连接和二段的提示线在is\_connect中绘制,三段连接提示线在is\_connect\_three中绘制,虽然实现了功能,但代码逻辑不清晰。